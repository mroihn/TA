\chapter{Analisis dan Perancangan}

\section{Analisis Masalah}

Berdasarkan kajian literatur, penerapan insentif blockchain pada ekosistem publikasi ilmiah menghadapi tiga tantangan fundamental yang saling berkaitan:

Pertama, terdapat ketidaksesuaian mekanisme validasi. Mayoritas model insentif terdesentralisasi saat ini mengandalkan \textit{crowd-voting} (suara terbanyak) untuk menentukan kualitas konten. Pendekatan ini tidak kompatibel dengan standar \textit{Open Journal System} (OJS) yang mewajibkan \textit{expert peer-review} tertutup (\textit{double-blind}) untuk menjaga objektifitas ilmiah. Ketiadaan mekanisme yang dapat menjembatani keputusan editorial tertutup (Off-chain) dengan eksekusi transparan di Blockchain (On-chain) menyebabkan sulitnya mengadopsi teknologi ini tanpa merombak total alur kerja akademik yang sudah mapan.

Kedua, terjadi dilema keberlanjutan pendanaan. Model insentif yang bergantung sepenuhnya pada biaya penulis (\textit{Submission Fee}) menciptakan hambatan partisipasi (\textit{Barrier to Entry}) bagi peneliti dengan dana terbatas. Sebaliknya, model yang hanya mengandalkan pencetakan token baru (\textit{Inflationary Minting}) berisiko tinggi menciptakan aset tanpa nilai riil (\textit{hyperinflation}) jika tidak diimbangi oleh permintaan pasar. Diperlukan mekanisme hibrida yang mampu menyeimbangkan beban biaya penulis dengan subsidi sistem secara proporsional.

Ketiga, risiko jual beli nilai akademik(SBT) akibat ketiadaan pemisahan fungsi aset. Dalam sistem konvensional maupun kripto sederhana, insentif finansial dan reputasi seringkali direpresentasikan oleh aset yang sama (likuid). Hal ini berbahaya karena reputasi seorang penelaah berpotensi diperjualbelikan layaknya komoditas pasar, yang pada akhirnya mendegradasi motivasi intrinsik (\textit{moral obligation}) menjadi motivasi transaksional semata. Diperlukan arsitektur manajemen aset yang secara tegas memisahkan insentif menjadi dua entitas: aset likuid untuk kebutuhan ekonomi dan aset terikat (\textit{soulbound}) untuk identitas reputasi.
\section{Rancangan Solusi}

Dari analisis masalah, diusulkan solusi mekanisme insentif hibrida dengan arsitektur interaksi sistem yang terdiri dari tiga komponen logika utama:

\begin{enumerate}
    \item \textbf{Formulasi Mekanisme Pendanaan (Funding Algorithm):} 
    Sistem menggunakan model probabilistik untuk menentukan total insentif ($I_{total}$) berdasarkan Skor Kualitas ($Q$). Mekanisme ini menggabungkan dana riil dari penulis ($F_{base}$) dan subsidi inflasi terkontrol ($M_{subsidy}$).
    
    Total insentif diformulasikan sebagai fungsi dari Skor Kualitas ($Q$) sebagai berikut:
    
    \begin{equation}
        I_{total}(Q) = 
        \begin{cases} 
        F_{base} + \underbrace{\left( k \cdot \frac{1}{V_{net}} \cdot (Q - T_{min}) \right)}_{M_{subsidy} \text{ (Komponen Minting)}} & \text{jika } Q > T_{min} \\
        0 & \text{jika } Q \leq T_{min}
        \end{cases}
    \end{equation}
    
    Dimana $F_{base}$ adalah biaya dasar penulis, $k$ adalah konstanta subsidi, dan $T_{min}$ adalah ambang batas kualitas.
    
    Analisis Komponen Minting ($M_{subsidy}$):
    Pencetakan token baru (\textit{Minting}) tidak dilakukan secara linier, melainkan diatur oleh algoritma dinamis dengan kondisi ketat untuk mencegah hiperinflasi:
    \begin{itemize}
        \item Kondisi Kelayakan ($Q > T_{min}$): Minting bersifat \textit{performance-based}. Sistem tidak akan mencetak token untuk ulasan di bawah standar kualitas, sehingga inflasi hanya terjadi untuk membayar "nilai tambah" (\textit{added value}).
        \item Elastisitas Suplai ($\frac{1}{V_{net}}$): Variabel $V_{net}$ didefinisikan sebagai Volume Submisi Jaringan per Epoch (rata-rata aktivitas dalam periode waktu tertentu, misal 30 hari), bukan total akumulatif sejak awal sistem. 
        
        % Mekanisme ini bertindak sebagai faktor deflasi dinamis yang responsif:
        % \begin{itemize}
        %     \item Saat aktivitas jaringan tinggi ($V_{net}$ besar), subsidi per artikel mengecil untuk mencegah banjir pasokan token (\textit{hyperinflation}).
        %     \item Definisi berbasis \textit{epoch} mencegah nilai pembagi membesar tanpa batas seiring waktu, yang dapat menyebabkan subsidi mendekati nol ($0$) di masa depan. Hal ini menjamin insentif tetap relevan bagi partisipan jangka panjang.
        % \end{itemize}
    \end{itemize}

    \item \textbf{Algoritma Distribusi dan Manajemen Aset (Execution Logic):}
    Implementasi teknis pada \textit{Smart Contract} mengatur aliran aset melalui dua fase atomik. Logika eksekusi memisahkan sumber dana (Transfer vs Minting) untuk transparansi audit.
    
    \begin{itemize}
        \item Fase 1 (Deposit Author): Memvalidasi komitmen penulis.
        
        {\footnotesize \setlength{\baselineskip}{0.8em} % Memadatkan jarak baris
        \begin{verbatim}
FUNCTION SubmitArticle(Fee):
    IF msg.value < Fee THEN REVERT("Dana Kurang")
    ELSE LOCK_TO_POOL(msg.value); EMIT Event("Deposit_OK")
        \end{verbatim}
        }
        \vspace{-0.5em}
        
        \item Fase 2 (Distribusi Hibrida): Dipicu oleh Editor. Memecah eksekusi menjadi dua vektor aset secara simultan (Ekonomi \& Reputasi).
        
        {\footnotesize \setlength{\baselineskip}{0.8em} % Memadatkan jarak baris
        \begin{verbatim}
FUNCTION DistributeReward(Reviewer, Score Q):
    IF Q <= T_min THEN RETURN 0
    Subsidi = (K * (Q - T_min) / Vol)          // Minting
        
    // 2. Eksekusi Atomik (Multi-Source Funding)
    TRANSFER_ERC20(Pool -> Reviewer, Base_Fee) // Dana Author
    MINT_ERC20(Reviewer, Subsidi)              // Inflasi Sistem
          
    // 3. Eksekusi Reputasi (Non-Transferable)
    MINT_SBT(Reviewer, Metadata={Score:Q})     // Vektor Reputasi
    SET SBT.Transferable = FALSE
        \end{verbatim}
        }
        \vspace{-0.5em}
    \end{itemize}
    
    Logika di atas memastikan bahwa \textit{Reviewer} menerima kompensasi finansial yang likuid dari dua sumber berbeda, namun reputasi akademik (SBT) terkunci permanen pada identitas mereka.
    
    \item \textbf{Arsitektur Interaksi Sistem (Oracle Protocol Design):}
    Untuk menjembatani kesenjangan antara OJS (\textit{Off-chain}) dan Blockchain (\textit{On-chain}) sesuai analisis masalah pertama, dirancang protokol komunikasi satu arah (\textit{One-Way Peg}) dengan peran Editor sebagai \textit{Trusted Oracle}. Alur interaksi dibagi menjadi tiga tahap sekuensial:
    
    \begin{itemize}
        \item Tahap 1: Inisiasi dan Validasi (Off-Chain Trigger):
        Proses dimulai di antarmuka OJS. Ketika Editor mengubah status artikel menjadi "Published", modul \textit{backend} OJS memaketkan data variabel (ID Reviewer, Skor $Q$, dan Alamat Wallet) menjadi struktur data JSON terstandarisasi (\textit{Payload}).
        
        \item Tahap 2: Penandatanganan Transaksi (Secure Bridging):
        Agar data yang dikirim ke Blockchain valid dan tidak dipalsukan, transaksi harus ditandatangani secara kriptografi (\textit{Digital Signature}) menggunakan \textit{Private Key} milik Editor atau Jurnal yang terdaftar sebagai \textit{Oracle} di \textit{Smart Contract}. Ini mencegah serangan injeksi data dari pihak luar.
        
        \item Tahap 3: Eksekusi dan Konfirmasi (On-Chain Settlement):
        Fungsi \textit{Smart Contract} menerima payload, memverifikasi tanda tangan pengirim (\textit{Signature Verification}), dan jika valid, langsung memicu fungsi distribusi aset (Poin 2). Hash transaksi kemudian dikembalikan ke OJS sebagai bukti pembayaran yang transparan.
    \end{itemize}
\end{enumerate}