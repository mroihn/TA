% ----- JUDUL BAB -----
\chapter{Analisis dan Perancangan}

\section{Analisis Masalah}

Sistem insentif dalam publikasi ilmiah saat ini menghadapi tantangan fundamental yang tidak dapat diselesaikan hanya dengan mendigitalkan proses pembayaran konvensional. Masalah utama berakar pada ketidakselarasan (\textit{misalignment}) antara motivasi ekonomi dan integritas akademik. Model insentif tradisional yang bergantung sepenuhnya pada kesukarelaan (\textit{volunteerism}) terbukti gagal membendung laju pertumbuhan naskah yang eksponensial, menyebabkan kelelahan (\textit{review fatigue}) dan penurunan kualitas telaah. Di sisi lain, upaya memberikan insentif finansial secara langsung sering kali terbentur pada risiko komodifikasi sains, di mana \textit{reviewer} bekerja mengejar kuantitas bayaran daripada kualitas masukan, sebuah fenomena yang dikenal sebagai \textit{Crowding Out Effect}.

Pendekatan berbasis \textit{blockchain} yang ada saat ini, seperti Steemit atau PubChain, menawarkan solusi desentralisasi namun membawa masalah baru ketika diterapkan dalam konteks akademik yang ketat. Model validasi berbasis \textit{voting} massa (\textit{crowd-voting}) yang diadopsi oleh platform tersebut sangat rentan terhadap manipulasi dan bias popularitas, serta bertentangan dengan prinsip \textit{double-blind review} yang mensyaratkan kerahasiaan dan kepakaran spesifik. Dalam OJS, validasi kualitas tidak bisa diserahkan kepada algoritma konsensus terbuka, melainkan harus tetap berada di bawah otorisasi Editor sebagai penanggung jawab mutu ilmiah.

Selain masalah validasi, keberlanjutan pendanaan (\textit{sustainability}) menjadi titik kegagalan fatal pada banyak model insentif sebelumnya. Sistem seperti Eureka yang bergantung sepenuhnya pada biaya penulis (\textit{submission fee}) berisiko mengalami \textit{deadlock} atau masalah \textit{Cold Start} ketika jumlah submisi rendah, menyebabkan ketiadaan dana untuk membayar \textit{reviewer}. Sebaliknya, model yang hanya mengandalkan inflasi token (\textit{minting}) tanpa adanya pemasukan nilai riil berisiko menciptakan token yang tidak bernilai (\textit{hyperinflation}). Oleh karena itu, diperlukan sebuah mekanisme ekonomi yang mampu menyeimbangkan pemasukan eksternal (fee) dengan subsidi sistem (\textit{minting}) untuk menjamin likuiditas insentif bagi \textit{reviewer} tanpa membebani penulis dengan biaya yang eksesif.

Terakhir, fragmentasi dalam pengelolaan aset digital menjadi hambatan teknis bagi adopsi di kalangan akademisi. \textit{Reviewer} saat ini tidak memiliki sarana terpadu untuk mengelola dua jenis hasil kerja mereka: imbalan finansial (aset likuid) dan reputasi akademik (aset non-likuid). Ketiadaan pemisahan yang tegas dalam arsitektur dompet digital (\textit{wallet}) berisiko mencampuradukkan motivasi, di mana reputasi bisa diperjualbelikan layaknya komoditas, yang pada akhirnya merusak kredibilitas sistem \textit{peer review} itu sendiri.

\section{Rancangan Solusi}

Berdasarkan analisis masalah di atas, penelitian ini mengusulkan rancangan solusi berupa Arsitektur Sistem Insentif Hibrida Terintegrasi yang dibangun di atas platform \textit{Open Journal Systems} (OJS). Solusi ini dirancang untuk menutup kelemahan masing-masing model terdahulu dengan menggabungkan mekanisme validasi terpusat (Editor) dengan distribusi insentif terdesentralisasi (\textit{Smart Contract}).

Untuk mengatasi masalah validasi dan integritas, sistem ini tidak mengadopsi mekanisme \textit{crowd-voting} maupun \textit{pure decentralized consensus}. Sebaliknya, solusi yang diajukan menempatkan Editor sebagai \textit{Oracle} Terpercaya dalam jaringan \textit{blockchain}. Dalam rancangan ini, \textit{Smart Contract} hanya akan mengeksekusi distribusi insentif setelah menerima sinyal validasi kualitas dari Editor. Pendekatan ini mempertahankan standar akademik \textit{double-blind review} sambil tetap memanfaatkan transparansi dan otomatisasi pembayaran yang ditawarkan oleh \textit{blockchain}.

Dari sisi ekonomi dan keberlanjutan, rancangan solusi ini menerapkan Model Pendanaan Sirkular (\textit{Circular Economy}) yang menggabungkan dua sumber likuiditas. Pertama, mekanisme \textit{Inflationary Minting} diadopsi untuk mencetak token subsidi secara otomatis pada setiap periode, menjamin ketersediaan ``gaji dasar'' bagi \textit{reviewer} meskipun tidak ada pemasukan dari penulis. Kedua, mekanisme \textit{Submission Fee} tetap diterapkan sebagai filter kualitas (\textit{spam barrier}) dan sumber nilai riil, namun dengan nominal yang terjangkau karena adanya subsidi silang dari \textit{minting}. Dana ini dikelola dalam sebuah \textit{Reward Pool} otonom yang mendistribusikan insentif secara proporsional berdasarkan bobot kualitas yang diberikan Editor.

Untuk aspek infrastruktur, penelitian ini memilih untuk tidak menggunakan \textit{Sidechain} publik maupun \textit{Mainnet} Ethereum secara langsung dikarenakan volatilitas biaya transaksi (\textit{gas fee}) yang tidak terprediksi dan risiko keamanan data pada jaringan publik. Sebagai gantinya, solusi dirancang menggunakan \textit{Consortium Blockchain} (seperti Hyperledger Besu) yang dikelola secara privat. Pendekatan ini memungkinkan transaksi insentif (\textit{minting} dan transfer) dilakukan dengan biaya nol (\textit{gas-free}) bagi pengguna akhir, serta menjamin privasi data \textit{reviewer} yang hanya dapat diakses oleh \textit{node-node} terotorisasi dalam jaringan kampus.

Dalam hal pengelolaan aset, rancangan solusi memanfaatkan \textit{MetaMask} sebagai antarmuka dompet digital standar yang dihubungkan ke jaringan privat. Pengembangan difokuskan pada logika \textit{Smart Contract} untuk memisahkan saldo token ERC-20 (insentif finansial) dan SBT (reputasi akademik) di dalam dompet tersebut, serta integrasi fungsi pemanggilan kontrak (\textit{contract calls}) pada kode sumber OJS yang sudah ada. Secara teknis, SBT diimplementasikan menggunakan standar \textit{Non-Fungible Token} (NFT) tipe ERC-721 yang dimodifikasi dengan menonaktifkan fitur transfer, menjadikannya sertifikat digital permanen yang melekat pada identitas \textit{reviewer}. Fitur \textit{Held-Back Vault} juga ditanamkan dalam logika kontrak pintar sebagai mekanisme jaminan mutu, di mana sebagian dana penulis dikunci sementara waktu untuk memitigasi risiko kecurangan pasca-publikasi.