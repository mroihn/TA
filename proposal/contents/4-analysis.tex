\chapter{Analisis dan Perancangan}

\section{Analisis Masalah}

Penelitian-penelitian terdahulu yang mengintegrasikan DLT dalam publikasi ilmiah memiliki keterbatasan. \textit{CryptSubmit} \cite{gipp2017cryptsubmit} berfokus pada \textit{timestamping} menggunakan \textit{blockchain} untuk memberikan bukti kepemilikan intelektual yang terverifikasi dan mencegah plagiarisme. Platform yang diusulkan \cite{Bestas2023Blockchain} mengimplementasikan \textit{autonomous assignment} dan \textit{reward distribution} menggunakan \textit{Ethereum smart contracts} untuk mengurangi bias dan mempercepat proses \textit{review}. Sementara itu, \textit{Eureka} \cite{niya2019blockchain} mengembangkan \textit{token-based incentive mechanism} untuk memberikan \textit{reward} bagi semua kontributor dalam proses publikasi, termasuk \textit{reviewer}, \textit{editor}, dan \textit{cited authors}.

Meskipun penelitian-penelitian tersebut menunjukkan potensi \textit{blockchain} dalam publikasi ilmiah, masih terdapat \textit{gap} signifikan dalam literatur. Sistem yang diusulkan umumnya hanya fokus pada aspek spesifik seperti \textit{timestamping}, \textit{assignment}, atau \textit{reward mechanism}, tanpa mengintegrasikan seluruh siklus publikasi secara \textit{end-to-end}. Oleh karena itu, penelitian ini bertujuan mengembangkan sistem publikasi ilmiah yang mengintegrasikan tiga karakteristik utama: \textit{open} untuk memastikan aksesibilitas penelitian tanpa hambatan, \textit{autonomous operation} melalui \textit{smart contracts} untuk mengurangi bias dan meningkatkan efisiensi, dan \textit{distributed ledger technology} untuk menjamin transparansi dan \textit{immutability} seluruh proses publikasi. Sistem terintegrasi ini diharapkan dapat mengatasi keterbatasan penelitian sebelumnya dengan menyediakan solusi komprehensif yang mencakup \textit{submission}, \textit{peer review}, \textit{reward distribution}, dan \textit{dissemination} dalam satu platform \textit{blockchain} yang kohesif.

\section{Rancangan Solusi}

Berdasarkan analisis masalah di atas, Tugas Akhir ini mengusulkan rancangan sistem publikasi ilmiah \textit{end-to-end} yang bersifat terbuka, otonom, dan berbasis DLT. Dengan mengadaptasi model alur kerja sistem publikasi konvensional \cite{Donovan2018Reflective} \cite{Rampelotto2023PeerReview}, alur kerja sistem yang diusulkan didefinisikan sebagai berikut:
\begin{enumerate}
    \item \textbf{\textit{Submission} :} 
    Penulis mengunggah naskah melalui \textit{platform} terdesentralisasi. Sistem secara otomatis melakukan \textit{hashing} terhadap dokumen dan mencatat \textit{timestamp} pada \textit{blockchain} sebagai bukti kepemilikan intelektual yang valid. \textit{Metadata} artikel diekstraksi dan \textit{hash} penyimpanannya dicatat pada \textit{distributed ledger}, sementara \textit{smart contract} merekam waktu pengajuan secara permanen untuk mencegah manipulasi data.
    \item \textbf{\textit{Initial Check} :} 
    \textit{Smart contract} mengeksekusi pemeriksaan otomatis terhadap format dokumen, kelengkapan \textit{metadata}, dan indikasi plagiarisme awal. Pada tahap ini, sistem juga memverifikasi saldo \textit{token}/\textit{cryptocurrency} penulis untuk memastikan ketersediaan biaya proses \textit{review}. Seluruh pemeriksaan teknis ini dilakukan tanpa intervensi manusia guna menjamin objektivitas prosedural.
    \item \textbf{\textit{Editorial Pre-check} :} 
    \textit{Smart contract} melakukan evaluasi kesesuaian naskah dengan cakupan (\textit{scope}) jurnal serta kualitas metodologi. Keputusan \textit{editor} (\textit{smart contract}) dicatat secara transparan di dalam \textit{blockchain}.
    \item \textbf{\textit{Peer Review} :} 
    \textit{Smart contract} menunjuk \textit{reviewer} secara otomatis berdasarkan riwayat kinerja yang tersimpan di \textit{blockchain}, dengan algoritma yang mencegah pemilihan rekan satu institusi atau kolaborator dekat untuk menghindari \textit{bias}. Laporan hasil tinjauan (\textit{review reports}) di-\textit{hash} dan disimpan secara \textit{immutable}. Sistem menerapkan model \textit{open peer review} untuk menjamin transparansi dan akuntabilitas proses penilaian.
    \item \textbf{\textit{Editorial Post-review} :} 
    Berdasarkan laporan \textit{reviewer}, \textit{smart contract} memfasilitasi pengambilan keputusan akhir (terima, revisi, atau tolak). Keputusan ini dicatat secara permanen di \textit{blockchain}, sehingga rekam jejak keputusan \textit{editorial} dapat diaudit sewaktu-waktu.
    \item \textbf{\textit{Revision and Resubmission} :} 
    Penulis melakukan revisi berdasarkan umpan balik \textit{reviewer}. Perubahan pada naskah dilacak menggunakan \textit{version control} berbasis \textit{blockchain}, di mana naskah revisi di-\textit{hash} ulang dan dibandingkan secara otomatis dengan versi sebelumnya. Hal ini memungkinkan \textit{reviewer} memverifikasi poin revisi secara efisien tanpa perlu membaca ulang keseluruhan naskah.
    \item \textbf{\textit{Final Acceptance & Reward Distribution} :} 
    Setelah status \textit{Final Acceptance} tercapai, \textit{smart contract} secara otomatis mengeksekusi distribusi insentif (\textit{token}) kepada \textit{reviewer}, \textit{editor}, dan penulis yang karyanya disitasi (\textit{cited authors}), memastikan kompensasi yang adil dan instan.
    \item \textbf{\textit{Publication} :} 
    Artikel yang diterbitkan disimpan secara penuh (\textit{full-text}) pada penyimpanan terdistribusi (seperti IPFS), sementara \textit{hash} referensinya dicatat di \textit{blockchain}.
    \item \textbf{\textit{Post-Publication} :} 
    Sistem mendukung \textit{open post-publication peer review}, memungkinkan komunitas ilmiah memberikan masukan berkelanjutan terhadap artikel yang telah terbit. Segala pembaruan dikelola melalui mekanisme \textit{versioning} berbasis \textit{blockchain}.
\end{enumerate}

\subsection{Kakas Pengembangan Sistem}
Pengembangan prototipe sistem publikasi ilmiah berbasis DLT akan diimplementasikan dengan kakas pemodelan sistem dan kakas pengembangan sebagai berikut:
\begin{enumerate}
    \item \textbf{\textit{Platform} DLT :} Tugas Akhir ini menggunakan \textit{blockchain} yang kompatibel dengan \textit{Ethereum}. Pemilihan ini didasarkan pada kematangan ekosistem \textit{smart contract} dan dukungan komunitas pengembang yang luas. Spesifikasi implementasi meliputi:  
    \item \textbf{\textit{Development Environment} :} Menggunakan \textit{Hardhat} untuk keperluan kompilasi, pengujian (\textit{testing}), dan \textit{deployment smart contract}
    \item \textbf{\textit{Smart Contract Language} :} Menggunakan bahasa pemrograman \textit{Solidity}.
    \item \textbf{\textit{Blockchain Network} :} Menggunakan \textit{public testnet Sepolia} (\textit{Ethereum}) sebagai lingkungan pengujian dan pengembangan untuk mensimulasikan kondisi jaringan yang realistis.
\end{enumerate}