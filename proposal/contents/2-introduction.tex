% ----- JUDUL BAB -----
\chapter{Pendahuluan}

\section{Latar Belakang}

Dalam beberapa tahun terakhir, sistem peer review telah menjadi komponen krusial dalam menjaga integritas dunia akademis dan ilmiah. Peer review adalah proses validasi di mana karya ilmiah dievaluasi oleh para ahli di bidang yang sama sebelum diterbitkan. Namun, ekosistem ini kini menghadapi krisis keberlanjutan karena laju pertumbuhan artikel yang dikirimkan tumbuh jauh lebih cepat daripada ketersediaan reviewer yang bersedia \cite{horta2024crisis}.

Menurut Horta et al. \cite{horta2024crisis}, dunia akademik saat ini berada di era "publish or perish", di mana terdapat dorongan eksternal yang kuat untuk mempublikasikan karya, namun hampir tidak ada insentif yang setara untuk melakukan review. Ketimpangan ini berhubungan erat dengan psikologis motivasi para reviewer. Secara historis, reviewer bekerja karena motivasi intrinsik (moral, tanggung jawab ilmiah, dan komitmen komunitas) \cite{squazzoni2013incentive}. Namun, upaya memberikan insentif material secara gegabah justru berpotensi memicu dampak negatif yang dikenal sebagai \textit{Crowding Effect}, di mana insentif eksternal dapat mengikis motivasi intrinsik seseorang \cite{squazzoni2013incentive}.

Untuk mengatasi krisis ini, diperlukan pendekatan teknologi baru yang mampu mengelola insentif secara transparan dan otomatis. Integrasi teknologi \textit{blockchain}, \textit{smart contract}, dan \textit{crypto wallet} menawarkan solusi potensial untuk menciptakan sistem insentif yang terdesentralisasi dan adil \cite{han2022blockchainincentivemechanisms}. Namun, tantangan utamanya bukan sekadar mendigitalkan pembayaran, melainkan bagaimana merancang arsitektur sistem yang tidak hanya memberikan imbalan moneter, tetapi juga mengakomodasi insentif non-moneter (reputasi). 

Berdasarkan studi pada platform komunitas \textit{blockchain} seperti Steemit, sistem insentif yang berkelanjutan harus bersifat multi-dimensi, yakni mampu menggabungkan aspek ekonomi, sosial, dan kepemilikan aset dengan seimbang \cite{liu2022steemit}. Oleh karena itu, diperlukan sebuah perancangan sistem yang mengintegrasikan mekanisme ekonomi sirkular dengan validasi kualitas terpusat ke dalam infrastruktur \textit{Open Journal Systems} (OJS), guna menciptakan ekosistem insentif yang otonom tanpa merusak integritas akademik.

\section{Rumusan Masalah}

Berdasarkan latar belakang di atas, rumusan masalah dalam penelitian ini adalah sebagai berikut:
\begin{enumerate}
    \item Bagaimana merancang mekanisme distribusi insentif hibrida pada \textit{Smart Contract} yang mampu menggabungkan sumber pendanaan dinamis untuk menjamin keberlanjutan dana insentif di OJS?
    \item Bagaimana merancang arsitektur integrasi sistem yang menghubungkan validasi kualitas tertutup oleh Editor di OJS (\textit{off-chain}) dengan eksekusi pembayaran otomatis dan manajemen aset reputasi pada \textit{crypto wallet} (\textit{on-chain})?
\end{enumerate}

\section{Tujuan dan Ukuran Keberhasilan Pencapaian}

Penelitian ini bertujuan untuk menjawab masalah yang dirumuskan di atas. Rincian tujuan beserta ukuran keberhasilannya adalah sebagai berikut:

\begin{enumerate}
    \item Mengimplementasikan algoritma insentif pada \textit{Smart Contract} yang secara otomatis mengelola alokasi dana dari biaya penulis dan subsidi sistem berdasarkan parameter validasi kualitas. Keberhasilan diukur dengan validasi fungsional logika kontrak pintar pada jaringan \textit{testnet}.
    \item Membangun prototipe modul integrasi pada OJS yang menghubungkan pemicu dari Editor dengan dompet kripto standar (\textit{MetaMask}) untuk pengelolaan aset likuid (token ERC-20) dan aset reputasi (\textit{Soulbound Token}). Keberhasilan diukur dengan demonstrasi alur kerja sistem dari submisi hingga pencairan insentif di sisi pengguna.
\end{enumerate}

\section{Batasan Masalah}

Penelitian ini memiliki batasan kajian sebagai berikut:
\begin{enumerate}
    \item Penelitian berfokus pada perancangan arsitektur perangkat lunak dan validasi fungsional prototipe sistem (logika \textit{smart contract} dan integrasi \textit{web3} pada OJS).
    \item Batasan kajian tidak mencakup validasi empiris atau studi perilaku jangka panjang untuk mengukur dampak psikologis (\textit{Crowding Effect}) dari model yang diusulkan terhadap populasi reviewer nyata.
    \item Implementasi \textit{blockchain} dilakukan pada lingkungan jaringan privat (seperti Hyperledger Besu atau simulasi lokal) dan tidak melibatkan transaksi aset kripto bernilai riil (\textit{Mainnet}).
    \item Antarmuka dompet digital menggunakan penyedia layanan standar industri (seperti MetaMask), bukan pengembangan aplikasi \textit{wallet} baru dari awal.
\end{enumerate}

\section{Metodologi}
Metodologi penelitian yang digunakan mengadopsi prinsip \textit{Design Science Research} (DSR), yang berfokus pada perancangan, implementasi, dan validasi artefak sistem. Sesuai dengan tujuan penelitian, tahapan tugas akhir ini adalah sebagai berikut:
\begin{enumerate}
    \item \textbf{Perancangan Mekanisme Insentif:}
    Tahap ini berfokus untuk menjawab rumusan masalah pertama. Kegiatan utamanya adalah merancang algoritma distribusi dana yang menggabungkan variabel biaya penulis dan inflasi sistem. \textit{Outcome} dari tahap ini adalah spesifikasi matematis dan logika bisnis \textit{smart contract}.
    
    \item \textbf{Perancangan Arsitektur Sistem Terintegrasi:}
    Tahap ini berfokus untuk menjawab rumusan masalah kedua. Kegiatan utamanya adalah merancang arsitektur teknis yang menghubungkan kode sumber OJS eksisting dengan jaringan \textit{blockchain}, serta pemetaan fungsi manajemen aset hibrida pada \textit{wallet}.
    
    \item \textbf{Implementasi Prototipe:} 
    Pada tahap ini, artefak yang dirancang pada tahap sebelumnya akan diimplementasikan ke dalam kode program. Kegiatan utamanya meliputi pengembangan \textit{Smart Contract} (Solidity) dan pengembangan modul integrasi pada antarmuka OJS.
    
    \item \textbf{Validasi Fungsional dan Demonstrasi:}
    Tahap ini berfokus untuk memvalidasi pencapaian tujuan penelitian. Validasi dilakukan melalui pengujian \textit{black-box} pada fungsi \textit{smart contract} dan demonstrasi skenario penggunaan (\textit{use case}) untuk membuktikan bahwa sistem dapat mendistribusikan insentif moneter dan reputasi secara akurat sesuai pemicu dari Editor.
\end{enumerate}