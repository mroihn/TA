% ------------------------------------------------------------------- %
%  INTRODUCTION CHAPTER
%  AUTHOR: Raden Francisco Trianto Bratadiningrat (@NoHaitch)
%  DATE: 2025-10-28
% ------------------------------------------------------------------- %

% ----- JUDUL BAB -----
\chapter{Pendahuluan}


% ----- LATAR BELAKANG -----
\section{Latar Belakang}

Sistem publikasi ilmiah modern sangat bergantung pada mekanisme \textit{peer review} sebagai dasar untuk menjaga kredibilitas dan kualitas pengetahuan ilmiah (\cite{Masic2016ZPeerReview}). 
\textit{Peer review} adalah proses penilaian manuskrip atau karya ilmiah yang akan dipublikasikan oleh para ahli sejawat di bidang terkait (\cite{drozdz2024peerreviewprocess}). 
Proses ini telah menjadi standar universal yang dipakai oleh sebagian besar jurnal ilmiah untuk menjaga mutu dan kredibilitas karya yang dipublikasikan (\cite{aliwatson2016peerreviewpublication}).

\textit{Peer review} adalah jantung dari proses-proses tidak hanya pada jurnal medis, tetapi juga di seluruh ilmu pengetahuan.
Mekanisme ini digunakan untuk mengalokasikan dana hibah, menerbitkan makalah, mempromosikan akademisi, dan bahkan memenangkan hadiah Nobel (\cite{Smith2006PeerReview}).
Namun, \textit{peer review} memiliki kekurangannya, seperti anonimitas, bias, transparansi, penyalahgunaan serta berbagai masalah lainnya (\cite{elguebaly2023peerreviewissues, tennant2020peerreviewissues, velterop2015peerreviewissues}): 

Dalam mengatasi kekurangannya, munculah model alternatif untuk \textit{peer review} guna mengatasi kelemahan-kelemahan tersebut, seperti \textit{Closed} dan \textit{Open Peer Review}, \textit{Single-Blind}, \textit{Double-Blind}, Hybrid dan berbagai model lainnya (\cite{drozdz2024peerreviewprocess}).
Namun, alternatif tersebut masih menghadapi tantangan dalam mencapai transparansi penuh, desentralisasi sistem, dan auditabilitas yang kuat, terutama dalam lingkungan publikasi ilmiah tradisional yang masih tersentralisasi (\cite{ware2008peer}).
Untuk mengatasi keterbatasan ini, teknologi Distributed Ledger Technology (DLT) menjadi pendekatan inovatif dengan sifat-sifat inheren seperti desentralisasi, immutabilitas catatan, dan transparansi transaksi (\cite{Morales2024BlockchainPeerReview}).   


% ----- RUMUSAN MASALAH -----
\section{Rumusan Masalah}

Studi ini bertujuan untuk merancang dan mengembangkan \textit{proof of concept} sistem publikasi ilmiah berbasis DLT dengan anonimitas \textit{double-blind}.
Sistem yang dikembangkan didasarkan pada model Open Peer Review(OPR), yang diadaptasikan dengan konsep anonimitas \textit{double-blind}. Pengembangan sistem bertujuan mendorong desentralisasi, transparansi, dan openness serta mengurangi bias dan penyalahgunaan.

% ----- KEBERHASILAN -----
\section{Tujuan dan Ukuran Keberhasilan Pencapaian}

Tujuan utama dari penelitian ini adalah merancang dan mengembangkan \textit{proof of concept} sistem publikasi ilmiah berbasis Distributed Ledger Technology (DLT) dengan mekanisme \textit{double-blind peer review}.

Secara khusus, tujuan penelitian ini meliputi:
\begin{enumerate}
    \item Mengembangkan sistem desentralisasi untuk proses peer review yang meningkatkan transparansi dan auditabilitas tanpa mengorbankan anonimitas penulis dan penelaah.
    \item Mengadaptasi dan mengimplementasikan model Open Peer Review (OPR) dengan anonimitas \textit{double-blind} dalam sistem yang didasarkan pada teknologi DLT.
    \item Menyediakan mekanisme smart contract untuk otomasi dan keamanan proses peer review di sistem terdistribusi.
    \item Melakukan pengujian \textit{proof of concept} menggunakan metode unit test dan simulasi alur proses peer review untuk memastikan fungsionalitas, keamanan, dan efektivitas sistem.
    \item Mengevaluasi kemampuan sistem dalam mengurangi bias, meningkatkan keadilan, dan menjaga integritas publikasi ilmiah melalui desentralisasi.
\end{enumerate}


% ----- BATASAN MASALAH -----
\section{Batasan Masalah}

Penelitian ini memiliki batasan sebagai berikut:
\begin{enumerate}
    \item Proses publikasi yang dibahas dalam penelitian ini terbatas pada tahapan mulai dari pengumpulan naskah ilmiah hingga persetujuan publikasi oleh editor. 
    \item Sistem insentif pada platform hanya dikaitkan dengan aktivitas dan kualitas penelaahan serta reputasi pengguna; aspek insentif lain seperti pendanaan penelitian atau penghargaan di luar proses publikasi tidak menjadi cakupan studi.
    \item Perancangan sistem tidak membahas detail spesifik seperti kriteria pemilihan penelaah, mekanisme kontrol waktu, atau aturan editorial spesifik yang berada di luar proses peer review inti; fokus studi hanya pada fungsi dasar sistem yang terintegrasi dalam model double-blind peer review berbasis DLT.
    \item Skema anonimitas dalam sistem memiliki keterbatasan, termasuk potensi deanonimisasi melalui konten naskah, meta-data, atau pola hubungan sosial antara penulis dan penelaah (\cite{Wisniewski2022Anonymity}).
    \item Studi ini hanya mengembangkan \textit{proof of concept} dan prototipe pada tingkat simulasi, sehingga tidak membahas integrasi atau penerapan pada platform publikasi jurnal sesungguhnya ataupun skala produksi.
\end{enumerate}

% ----- METODOLOGI -----
\section{Metodologi}

Metodologi yang akan digunakan pada studi ini adalah metodologi berbasis model Waterfall (\cite{Ruparelia2010SDLCModels}):

\begin{enumerate}
    \item Pendefinisian spesifikasi dan kebutuhan sistem
    \item Analisis persoalan
    \item Perancangan sistem
    \item Pengembangan sistem
    \item Pengujian sistem
\end{enumerate}
