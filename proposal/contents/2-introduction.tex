% ----- JUDUL BAB -----
\chapter{Pendahuluan}

\section{Latar Belakang}

Sistem publikasi ilmiah tradisional telah menjadi tulang punggung penyebaran pengetahuan selama berabad-abad, dimulai dari kemunculan jurnal akademis pertama seperti \textit{Journal des Sçavans} dan \textit{Philosophical Transactions} pada tahun 1665 \cite{Banks2018Thoughts}. Sistem ini melibatkan proses \textit{peer review} yang dikelola oleh penerbit terpusat, di mana \textit{editor} dan \textit{reviewer} mengevaluasi naskah sebelum publikasi. Namun, model ini dinilai bermasalah karena membebankan biaya tinggi, menghilangkan hak cipta penulis, serta kurang transparan dan minim insentif bagi kontributor seperti \textit{reviewer} \cite{Bestas2023Blockchain}. Selain rentan terhadap bias, proses ini sangat lambat dengan durasi peninjauan hingga 12 bulan \cite{Banks2018Thoughts} dan mengalami krisis kredibilitas akibat rendahnya tingkat reproduktifitas penelitian, yang pada akhirnya menghambat validasi dan penyebaran ilmu pengetahuan \cite{niya2019blockchain,Bestas2023Blockchain}.

\textit{Open Access Publication} (OA) hadir sebagai solusi untuk mengatasi hambatan aksesibilitas dan biaya dalam penyebaran karya ilmiah. Mengacu pada \cite{Suber2012OpenAccess}, OA didefinisikan sebagai literatur digital yang tersedia secara daring, bebas biaya, serta terbebas dari sebagian besar batasan hak cipta dan lisensi, sehingga memungkinkan pemanfaatan hasil penelitian tanpa hambatan finansial maupun legal. Dalam ekosistem \textit{Open Science}, konsep ini diperluas melalui penerapan \textit{Open Peer Review} (OPR). \cite{ross-hellauer2017} mendeskripsikan OPR sebagai istilah payung bagi berbagai model ulasan, seperti saling membuka identitas penulis dan penelaah, mempublikasikan laporan hasil ulasan, serta membuka partisipasi bagi komunitas yang lebih luas. Praktik ini bertujuan meningkatkan transparansi dan akuntabilitas proses evaluasi, sekaligus memberikan rekognisi yang layak atas kontribusi para penelaah. Lebih lanjut, \cite{Gasparyan2019JKMS} menegaskan bahwa OA memerlukan pendekatan komprehensif yang tidak hanya terbatas pada penyediaan akses gratis, melainkan juga mencakup peningkatan transparansi dalam proses ulasan, kepatuhan terhadap standar etika, serta diversifikasi metrik evaluasi untuk mengukur dampak penelitian secara lebih akurat.

Sistem publikasi ilmiah berbasis \textit{blockchain} yang diusulkan oleh \cite{Bestas2023Blockchain} memanfaatkan \textit{smart contracts} untuk menjalankan protokol kesepakatan secara otonom, sehingga meminimalkan intervensi manusia dalam pengambilan keputusan administratif. Melalui kode yang dieksekusi otomatis di jaringan \textit{Ethereum}, sistem ini mampu menyeleksi \textit{editor} dan penelaah yang paling relevan serta mendistribusikan insentif (\textit{rewards}) secara langsung kepada seluruh kontributor tanpa perantara. Mekanisme otonom ini secara signifikan mengurangi bias dan hambatan birokrasi manual, menciptakan proses evaluasi yang lebih cepat, transparan, dan objektif.

\textit{Distributed Ledger Technology} (DLT), khususnya \textit{blockchain}, menawarkan infrastruktur teknis untuk mewujudkan sistem publikasi yang \textit{open} dan \textit{autonomous} melalui pencatatan yang \textit{immutable}, transparansi penuh, dan eksekusi \textit{smart contract} yang \textit{trustless}. Beberapa penelitian telah mengeksplorasi penerapan \textit{blockchain} dalam publikasi ilmiah dengan fokus yang berbeda-beda. \textit{CryptSubmit} \cite{gipp2017cryptsubmit} berfokus pada \textit{timestamping} menggunakan \textit{Bitcoin blockchain} untuk memberikan bukti kepemilikan intelektual yang terverifikasi dan mencegah plagiarisme. Platform yang diusulkan \cite{Bestas2023Blockchain} mengimplementasikan \textit{autonomous assignment} dan \textit{reward distribution} menggunakan \textit{Ethereum smart contracts} untuk mengurangi bias dan mempercepat proses \textit{review}. Sementara itu, \textit{Eureka} \cite{niya2019blockchain} mengembangkan \textit{token-based incentive mechanism} untuk memberikan \textit{reward} bagi semua kontributor dalam proses publikasi, termasuk \textit{reviewer}, \textit{editor}, dan \textit{cited authors}.

Meskipun penelitian-penelitian tersebut menunjukkan potensi \textit{blockchain} dalam publikasi ilmiah, masih terdapat \textit{gap} signifikan dalam literatur. Sistem yang diusulkan umumnya hanya fokus pada aspek spesifik seperti \textit{timestamping}, \textit{assignment}, atau \textit{reward mechanism}, tanpa mengintegrasikan seluruh siklus publikasi secara \textit{end-to-end}. Oleh karena itu, Tugas Akhir ini bertujuan mengembangkan sistem publikasi ilmiah yang mengintegrasikan tiga karakteristik utama: \textit{open} untuk memastikan aksesibilitas penelitian tanpa hambatan, \textit{autonomous operation} melalui \textit{smart contracts} untuk mengurangi bias dan meningkatkan efisiensi, dan \textit{distributed ledger technology} untuk menjamin transparansi dan \textit{immutability} seluruh proses publikasi. Sistem terintegrasi ini diharapkan dapat mengatasi keterbatasan penelitian sebelumnya dengan menyediakan solusi komprehensif yang mencakup \textit{submission}, \textit{peer review}, \textit{reward distribution}, dan \textit{dissemination} dalam satu platform \textit{blockchain} yang kohesif.

\section{Rumusan Masalah}

Studi ini bertujuan untuk mengembangkan \textit{prototipe} sistem publikasi berbasis \textit{Distributed Ledger Technology} (DLT) yang mengintegrasikan \textit{workflow end-to-end} publikasi ilmiah dari \textit{submission}, \textit{peer review}, hingga \textit{publishing} dengan mengintegrasikan tiga karakteristik utama: \textit{open} untuk memastikan aksesibilitas penelitian tanpa hambatan, \textit{autonomous operation} melalui \textit{smart contracts} untuk mengurangi bias dan meningkatkan efisiensi, dan \textit{distributed ledger technology} untuk menjamin transparansi dan \textit{immutability} seluruh proses publikasi

\section{Tujuan dan Ukuran Keberhasilan Pencapaian}

Berdasarkan rumusan masalah yang telah diusulkan, tujuan dari tugas akhir ini adalah mengembangkan \textit{prototipe} publikasi ilmiah \textit{end-to-end} berbasis \textit{Distributed Ledger Technology} (DLT) yang menyatukan karakteristik \textit{open}, \textit{autonomous}, serta \textit{distributed}. Keberhasilan pengembangan ini diukur berdasarkan validitas eksekusi otomatis fungsi-fungsi sistem, terciptanya jejak audit digital yang transparan dan aman pada jaringan \textit{blockchain}, serta terealisasinya mekanisme publikasi yang sepenuhnya objektif dan bebas diakses oleh publik.

\section{Batasan Masalah}

Penelitian ini memiliki batasan kajian sebagai berikut:
\begin{enumerate}
    \item \textbf{Mekanisme Modul:} Modul - Modul  yang diimplementasikan pada tugas akhir ini masih bersifat \textit{prototipe} atau \textit{proof-of-concept}. 
    \item \textbf{Batasan Pengujian:} Pengujian fungsionalitas \textit{prototipe} akan dilakukan secara terbatas menggunakan metode uji kasus untuk memvalidasi alur kerja \textit{prototipe}. Penelitian ini tidak mencakup pengujian kinerja terhadap beban tinggi atau analisis keamanan terhadap kerentanan \textit{smart contract}.
\end{enumerate}

\section{Metodologi}
Penelitian ini menggunakan pendekatan metode pengembangan perangkat lunak Waterfall, Tahapan penelitian adalah sebagai berikut:
\begin{enumerate}
    \item \textbf{Analisis Kebutuhan (\textit{Requirement Analysis}):}
    Tahap ini berfokus pada analisis kebutuhan fungsional sistem publikasi.
    
    \item \textbf{Perancangan Sistem (\textit{System Design}):}
    Tahap ini menerjemahkan kebutuhan ke dalam rancangan arsitektur teknis. Kegiatan utamanya meliputi perancangan \textit{DApps} dan perancangan struktur data pada \textit{Smart Contract}. \textit{Outcome} dari tahap ini adalah dokumen spesifikasi desain sistem publikasi.
    
    \item \textbf{Implementasi (\textit{Implementation}):}
    Tahap penerjemahan desain ke dalam kode program. Fokus kegiatan adalah pengembangan \textit{DApps}, serta pembuatan skrip pengujian.
    
    \item \textbf{Pengujian (\textit{Testing}):}
    Tahap verifikasi untuk memastikan implementasi berjalan sesuai tujuan. Pengujian dilakukan dengan metode \textit{Unit Testing} dan simulasi skenario transaksi pada jaringan lokal (\textit{Local Testnet}).
\end{enumerate}