
% ---------- JUDUL BAB ----------
\chapter{Kajian Pustaka}

\section{Sistem Publikasi Ilmiah}
Sistem publikasi ilmiah merupakan mekanisme fundamental dalam penyebaran pengetahuan yang telah berevolusi sejak penerbitan jurnal akademik pertama, \textit{Journal des Sçavans} dan \textit{Philosophical Transactions} pada tahun 1665. \cite{Banks2018Thoughts} menjelaskan bahwa motivasi awal publikasi ilmiah bergeser dari sekadar membangun reputasi pribadi dan berbagi temuan menjadi tuntutan karier yang mendesak di era modern. Secara operasional, \cite{Donovan2018Reflective} menguraikan bahwa sebuah karya ilmiah melalui serangkaian tahapan ketat mulai dari persiapan, penyerahan (\textit{submission}), peninjauan (\textit{review}), revisi, hingga penerimaan dan publikasi akhir. Di dalam alur ini, proses \textit{peer review} memegang peranan sentral sebagai mekanisme kendali mutu untuk memastikan validitas dan kebaruan sebuah temuan sebelum disebarluaskan kepada komunitas global \cite{Rampelotto2023PeerReview}.

Meskipun sistem konvensional ini telah mapan, berbagai literatur menyoroti tantangan signifikan yang menghambat efisiensi dan keadilan ekosistem publikasi saat ini. \cite{Suber2012OpenAccess} mengidentifikasi adanya "krisis harga" di mana biaya berlangganan jurnal meningkat jauh melampaui inflasi, menciptakan hambatan akses bagi peneliti dan institusi. Selain faktor biaya, masalah transparansi dan insentif juga menjadi sorotan utama. \cite{Rampelotto2023PeerReview} mencatat bahwa kurangnya pengakuan terhadap kontribusi \textit{reviewer} sering kali menyebabkan kelangkaan peninjau yang berkualitas, yang pada gilirannya memperlambat proses \textit{editorial}. Lebih jauh, model penerbitan terpusat yang ada saat ini dinilai rentan terhadap \textit{bias editorial} dan ketidakefisienan proses yang dapat menghambat \textit{diseminasi} ilmu pengetahuan secara tepat waktu.

\section {Karakteristik Open dan Autonomous pada Publikasi Ilmiah}
Untuk mengatasi keterbatasan sistem tradisional yang tertutup dan terpusat, diperlukan pergeseran paradigma menuju model yang lebih modern. Penelitian ini mengusulkan integrasi dua karakteristik fundamental: \textit{Open} (terbuka) dan \textit{Autonomous} (otonom). Karakteristik \textit{Open} menjamin aksesibilitas dan transparansi proses, sementara karakteristik \textit{Autonomous} memastikan keberlanjutan sistem tanpa ketergantungan pada otoritas tunggal melalui otomatisasi teknologi.

\subsection{\textit{Open Access Publication} dan \textit{Open Peer Review}}
Konsep keterbukaan dalam publikasi ilmiah diwujudkan melalui \textit{Open Access} (\textit{OA}).\cite{Suber2012OpenAccess} mendefinisikan \textit{OA} sebagai literatur digital yang tersedia secara daring, gratis, dan bebas dari sebagian besar hambatan hak cipta serta lisensi. Ia membedakan dua jenis utama \textit{OA}: \textit{Gratis OA} yang hanya menghilangkan hambatan harga, dan \textit{Libre OA} yang juga menghilangkan hambatan izin penggunaan kembali. Sejalan dengan ini, \cite{Gasparyan2019JKMS} menekankan pentingnya inisiatif global seperti \textit{Plan S} yang mewajibkan penelitian yang didanai publik untuk segera tersedia dalam akses terbuka, serta peran \textit{platform pengindeksan} dalam meningkatkan visibilitas dan jangkauan karya ilmiah secara global.

Selain akses terhadap artikel, keterbukaan juga diterapkan pada proses evaluasi melalui \textit{Open Peer Review} (\textit{OPR}).\cite{ross-hellauer2017} dalam tinjauan sistematisnya mengungkapkan bahwa meskipun definisi \textit{OPR} beragam, terdapat ciri-ciri utama yang membedakannya dari \textit{blind review} tradisional. Karakteristik tersebut meliputi \textit{Open Identities}, di mana penulis dan peninjau saling mengetahui identitas masing-masing untuk meningkatkan akuntabilitas, dan \textit{Open Reports}, di mana laporan peninjauan dipublikasikan berdampingan dengan artikel. Penerapan \textit{OPR} bertujuan untuk meningkatkan kualitas ulasan, mengurangi \textit{bias}, dan memfasilitasi diskusi konstruktif dalam komunitas ilmiah.

\subsection{\textit{Autonomous Operation} melalui \textit{Smart Contracts}}
Otonomi dalam sistem publikasi dapat dicapai melalui pemanfaatan \textit{Smart Contracts}. \cite{Zheng2017BlockchainOverview} mendefinisikan \textit{Smart Contracts} sebagai protokol transaksi terkomputerisasi yang mengeksekusi ketentuan kontrak secara otomatis ketika kondisi yang telah ditentukan terpenuhi. Berjalan di atas teknologi \textit{blockchain}, kontrak pintar ini memungkinkan kode untuk berjalan persis seperti yang diprogram tanpa kemungkinan penyensoran, \textit{downtime}, atau campur tangan pihak ketiga, sehingga menciptakan lingkungan yang \textit{trustless} (tidak memerlukan kepercayaan pada perantara manusia).

Penerapan konsep ini dalam publikasi ilmiah memungkinkan otomatisasi sebagian besar proses \textit{editorial} yang sebelumnya dilakukan secara manual oleh penerbit. \cite{Bestas2023Blockchain} mendemonstrasikan bagaimana \textit{Smart Contracts} dapat menangani alur kerja mulai dari penyerahan naskah, penunjukan \textit{reviewer}, hingga keputusan penerbitan secara terdesentralisasi. Dengan menghilangkan perantara terpusat, sistem otonom ini tidak hanya mempercepat waktu publikasi tetapi juga mengurangi biaya operasional secara signifikan dan menjamin integritas data melalui mekanisme eksekusi yang transparan dan tidak dapat diubah.

\section{\textit{Distributed Ledger Technology}}
	\textit{Distributed Ledger Technology} (\textit{DLT}) merupakan fondasi teknologi yang memungkinkan terciptanya sistem yang terdesentralisasi. \cite{RomeroUgarte2018DLT} menjelaskan \textit{DLT} sebagai basis data di mana terdapat banyak salinan identik yang didistribusikan ke berbagai partisipan (\textit{node}) dalam jaringan \textit{Peer-to-Peer} (\textit{P2P}). Berbeda dengan basis data terpusat, \textit{DLT} menggunakan mekanisme konsensus kriptografis untuk memvalidasi perubahan data, sehingga memastikan bahwa semua salinan buku besar (\textit{ledger}) tetap sinkron dan akurat tanpa memerlukan administrator pusat.

\section{\textit{Related Works}}
Berbagai penelitian terdahulu telah mengeksplorasi penggunaan teknologi \textit{blockchain} dalam aspek spesifik publikasi ilmiah. \cite{gipp2017cryptsubmit} mengembangkan solusi bernama \textit{"CryptSubmit"} yang memanfaatkan \textit{blockchain Bitcoin} untuk memberikan stempel waktu (\textit{timestamping}) yang aman pada naskah saat diserahkan. Fokus utama penelitian ini adalah melindungi hak kekayaan intelektual penulis dari potensi plagiarisme oleh \textit{reviewer} atau \textit{editor} yang tidak jujur sebelum naskah dipublikasikan, namun belum mencakup keseluruhan proses penerbitan.

Pendekatan yang lebih komprehensif diajukan oleh \cite{niya2019blockchain} melalui \textit{platform "Eureka"}. Penelitian ini mengusulkan sistem penerbitan berbasis \textit{blockchain} yang memberikan insentif berupa \textit{token kripto} kepada penulis dan \textit{reviewer} untuk meningkatkan kualitas dan efisiensi proses peninjauan. \textit{Eureka} berfokus pada penyelesaian masalah krisis reproduktifitas dalam sains dengan mendorong publikasi data mentah dan hasil studi negatif yang sering kali diabaikan oleh jurnal tradisional.

Selanjutnya, \cite{Bestas2023Blockchain} merancang sistem publikasi ilmiah berbasis \textit{Ethereum} yang memanfaatkan \textit{Smart Contracts} untuk mengelola reputasi dan alur kerja \textit{editorial}. Solusi ini menekankan pada penghapusan biaya tinggi yang dibebankan oleh penerbit komersial dan mengembalikan kontrol hak cipta kepada penulis. Sistem ini juga memperkenalkan mekanisme \textit{pemungutan suara berbasis komunitas} untuk menentukan penerimaan artikel.

Meskipun penelitian-penelitian di atas telah memberikan kontribusi signifikan, terdapat kesenjangan (\textit{gap}) yang perlu diisi. Solusi seperti \cite{gipp2017cryptsubmit} masih terbatas pada aspek keamanan dokumen (\textit{timestamping}) dan belum menyentuh manajemen alur kerja. Di sisi lain, \textit{platform} seperti \textit{Eureka} \cite{niya2019blockchain} dan usulan \cite{Bestas2023Blockchain} lebih berfokus pada insentif ekonomi dan \textit{desentralisasi} umum. Penelitian ini hadir untuk mengisi celah tersebut dengan mengintegrasikan karakteristik \textit{"Open"} secara spesifik ke dalam arsitektur yang otonom, sehingga menciptakan sebuah prototipe sistem yang tidak hanya terdesentralisasi dan aman, tetapi juga transparan dalam evaluasi kualitas ilmiah secara \textit{end-to-end}.






