
% ---------- JUDUL BAB ----------
\chapter{Kajian Pustaka}

% -manusia butuh insentif (Material, Solidaritas, Tujuan).-
\section{Teori Dasar Sistem Insentif}
Setiap organisasi yang fungsional pasti bergantung pada kontribusi dari para anggotanya. 
Individu adalah "faktor strategis dasar" dalam organisasi sehingga mereka harus didorong atau "dibujuk" agar ada keinginan 
untuk bekerja sama. Jika tidak ada, maka tidak akan kerja sama dan dorongan inilah yang disebut sebagai insentif \cite{clark1961incentivesystems}.

Karena itu, diusulkan sebuah teori organisasi yang berdasar pada klasifikasi sistem insentif yang digunakan
untuk memotivasi kontributor. Berdasarkan jenis insentif utama, dibagi menjadi 3 kategori insentif yaitu: material, solidaritas dan tujuan \cite{clark1961incentivesystems}.

Definisi dari ketiga kategori insentif tersebut adalah sebagai berikut\cite{clark1961incentivesystems}:
\begin{itemize}
    \item Insentif Material Ini adalah imbalan yang bersifat nyata (tangible) dan memiliki nilai moneter atau dapat dengan 
    mudah diterjemahkan ke dalam nilai moneter. Contoh paling umum dari insentif material adalah uang dalam bentuk upah, gaji, 
    atau tunjangan-tunjangan lain yang dapat diukur secara finansial
    \item Insentif Solidaritas Ini adalah imbalan yang bersifat tidak nyata (intangible) dan tidak memiliki
    nilai moneter. Berbeda dengan insentif lain, insentif solidaritas berasal dari tindakan berasosiasi itu sendiri.
    Ini mencakup imbalan seperti sosialisasi, rasa kekeluargaan, rasa keanggotaan dan identifikasi kelompok, serta status yang 
    diperoleh dari keanggotaan tersebut. Ciri khas insentif ini adalah ia cenderung "independen dari tujuan akhir" organisasi.
    \item Insentif Tujuan Ini juga bersifat tidak nyata (intangible), namun tidak seperti solidaritas,
    insentif ini berasal dari tujuan yang dinyatakan oleh asosiasi. Insentif ini ditemukan dalam "tujuan suprapersonal" organisasi,
    seperti memperjuangkan suatu legislasi, memberantas korupsi, atau menyebarkan informasi. Ciri khasnya adalah
    insentif ini "tidak dapat dipisahkan" dari tujuan akhir yang sedang diperjuangkan.
\end{itemize}


% -insentif "Solidaritas/Tujuan" (sukarela) sudah overload, tapi insentif "Material" murni (uang) malah merusak motivasi. Kita butuh sistem baru.-
\section{Krisis Insentif dalam Peer Review Akademik}
Bagian ini mengidentifikasi akar masalah dalam sistem peer review saat ini, dengan menganalisis dua kegagalan utama:
kegagalan sistem non-insentif saat ini dan kegagalan solusi insentif material yang sederhana.

Krisis peer review yang terjadi saat ini berakar pada overburden dan fatigue yang dialami oleh para reviewer.
Penyebab utama dari krisis ini sebagai ketidakseimbangan struktural yang fundamental dalam dunia
akademis kontemporer. Dalam era "publish or perish", terdapat insentif yang jelas untuk mempublikasikan,
namun di sisi lain terdapat sedikit atau tidak ada insentif untuk mereview\cite{horta2024crisis}.

Akibatnya, aktivitas me-review bisa dikatakan sebagai pekerjaan tambahan yang tidak diakui secara formal dalam jenjang karir dan bergantung hampir
seluruhnya pada kerja sukarela. Sistem ini hanya ditopang oleh motivasi yang didasarkan pada kewajiban atau pelayanan
kepada komunitas ilmiah yang setara dengan insentif solidaritas dan tujuan. Namun, volume publikasi ilmiah yang terus tumbuh
membuktikan bahwa motivasi intrinsik ini tidak lagi mencukupi untuk menopang permintaan.

Jika sistem non-insentif saat ini gagal, solusi yang paling intuitif adalah memberikan insentif material.
Namun, pendekatan ini terbukti kontraproduktif. Editor jurnal menghadapi tantangan konstan untuk meningkatkan keandalan
dan komitmen dari reviewer. Hubungan yang problematik ini sebagai masalah principal-agent, 
di mana terdapat kesulitan mendasar bagi editor (principal) untuk menyelaraskan kepentingan reviewer (agen) dengan tujuan jurnal \cite{squazzoni2013incentive}. Untuk mengujinya,
sebuah studi eksperimental krusial secara spesifik menguji dampak pemberian insentif material terhadap kualitas peer review

Berlawanan dengan ekspektasi, temuan mereka menunjukkan bahwa menawarkan imbalan material cenderung menurunkan kualitas dan efisiensi 
dari proses review. Secara spesifik, skema insentif tetap/fixed terbukti menjadi model terburuk dalam mempromosikan kerja sama. Skema insentif
tetap ini menghasilkan Cooperation Index (CI) yang secara signifikan lebih rendah dibandingkan dengan skema tanpa pemberian insentif, yang justru mencatatkan CI tertinggi.
Fenomena di mana insegntif eksternal justru mengikis atau melemahkan motivasi moral intrinsik ini dikenal sebagai
Motivation Crowding Theory. Temuan ini menunjukkan dilema inti: sistem tanpa insentif sedang runtuh karena beban berlebih,
sementara sistem dengan insentif material sederhana \cite{squazzoni2013incentive} juga gagal karena merusak motivasi intrinsik.\cite{horta2024crisis}


\section{Mekanisme Insentif di Jaringan P2P}
Berhubungan dengan tantangan didalam sebuah organisasi yang telah dibahas sebelumnya, jaringan terdesentralisasi atau P2P 
menghadapi sebuah masalah yang cukup serupa terkait aspek partisipasi. Tanpa sebuah mekanisme insentif, jaringan P2P rentant
terhadap perilaku egois dimana partisipan hanya mengambil sumber daya tanpa adanya kontribusi \cite{ihle2023incentivemechanisms}.
Karena itu dibutuhkan sebuah mekanisme yang berfungsi menyelaraskan kepentingan individu dengan tujuan jaringan tersebut melalui tekanan
unsur insentif berupa hadiah dan juga unsur hukuman.

Berdasarkan tinjauan sistematis terhadap literatur P2P, mekanisme insentif dapat diklasifikasikan ke dalam tiga kategori 
utama yang memiliki korelasi dengan teori insentif dasar: insentif moneter, reputasi, dan layanan
\begin{itemize}
    \item Insentif Moneter: Jenis insentif yang memotivasi partisipasi dengan mendistribusikan kredit atau mata uang digital
    Dalam konteks P2P sering kali diimplementasikan dalam bentuk lelang. Namun penerapan insentif ini menimbulkan tantangan
    dalam jaringan P2P yaitu double-spending, namun kini dapat diatasi dengan teknologi DLT (Distributed Ledger) tanpa memerlukan
    otoritas pusat \cite{ihle2023incentivemechanisms}.)
    \item Insentif Reputasi: Jenis insentif yang memotivasi partisipasi dengan memberikan penghargaan sosial atau status berupa reputasi.
    Nilai reputasi bersifat subjektif, dinamis dan tidak dapat ditransfer serta melekat pada identitas seseorang. Mekanisme ini sangat cocok
    untuk memitigasi efek negatif dari insentif material murni karena partisipan dimotivasi langsung oleh status sosial dan kepercayaan dalam jaringan.
    \cite{ihle2023incentivemechanisms}
    \item Insentif Layanan: Mekanisme ini memberikan imbalan berupa akses instan ke sumber daya 
    (seperti bandwidth atau komputasi) sebagai bentuk timbal balik langsung (tit-for-tat) tanpa perlu melacak sejarah 
    transaksi yang panjang\cite{ihle2023incentivemechanisms}
\end{itemize}

Penerapan insentif dalam jaringan P2P harus didesain sedemikian rupa agar 
terhadap berbagai macam bentuk serangan meliputi Sybil Attack dan kolusi antar partisipan\cite{ihle2023incentivemechanisms}.
Dengan memanfaatkan teknologi Blockchain dan \textit{Smart Contract} memungkinkan "single source of truth" yang transparan
untuk mengelola insentif secara adil tanpa ada interverensi pihak lain.



\section{Blockchain, Smart Contract, dan Manajemen Aset}

Untuk mengimplementasikan mekanisme insentif yang efektif dalam lingkungan \textit{peer-to-peer} (seperti yang dibahas pada Sub-bab II.3), diperlukan infrastruktur teknologi yang mampu menjamin kepercayaan, transparansi, dan otomatisasi tanpa perantara. Bagian ini menguraikan tiga komponen teknologi fundamental yang menjadi landasan sistem yang diusulkan: Blockchain, Smart Contract, dan Arsitektur Crypto Wallet.

\subsection{Distributed Ledger Technology}
Blockchain berfungsi sebagai \textit{distributed ledger} (buku besar terdistribusi) yang mencatat seluruh riwayat transaksi dan reputasi secara \textit{immutable} (tidak dapat diubah). Dalam konteks publikasi ilmiah, sifat \textit{tamper-proof} ini krusial untuk mencegah manipulasi skor ulasan atau riwayat pembayaran, masalah yang sering terjadi pada sistem terpusat konvensional.

Namun, implementasi langsung pada jaringan utama (\textit{mainnet}) sering terkendala biaya transaksi (\textit{gas fee}) yang tinggi. Solusi untuk tantangan ini adalah teknologi \textit{Pegged Sidechains}. Mekanisme ini memungkinkan aset dipindahkan dari rantai utama ke rantai samping (\textit{sidechain}) melalui proses penguncian (\textit{locking}) dan pencetakan ulang (\textit{minting}), memungkinkan efisiensi biaya tanpa mengorbankan keamanan aset \cite{back2014pegged}.

\subsection{Smart Contract}
\textit{Smart Contract} adalah protokol transaksi yang mengeksekusi ketentuan kontrak secara otomatis ketika kondisi yang telah diprogram terpenuhi. Teknologi ini mentransformasi proses insentif dari yang sebelumnya manual dan bergantung pada kepercayaan terhadap admin jurnal, menjadi sistem yang deterministik \cite{khan2021smartcontracts}.

Dalam arsitektur sistem insentif, \textit{Smart Contract} memegang peran vital sebagai \textit{Escrow} Otomatis (rekening bersama). Dana insentif dari penulis atau institusi dikunci dalam kontrak dan hanya akan didistribusikan secara otomatis kepada \textit{reviewer} saat validasi kualitas terpenuhi, menghilangkan risiko wanprestasi pembayaran \cite{xuan2020datasharing}.

\subsection{Manajemen Aset dalam Wallet Crypto}
Komponen terakhir adalah \textit{Crypto Wallet} yang berfungsi lebih dari sekadar penyimpan mata uang. Dalam desain insentif multi-dimensi, dompet digital harus mampu membedakan dua jenis aset digital yang merepresentasikan insentif moneter dan non-moneter:

\begin{itemize}
    \item Fungible Tokens (Utility): Token standar (seperti ERC-20) yang bersifat likuid, dapat dipertukarkan, dan berfungsi sebagai alat pembayaran insentif finansial \cite{niya2019blockchain}.
    
    \item Soulbound Tokens (SBT): Jenis token khusus yang terikat secara permanen pada alamat dompet pemilik dan tidak dapat dipindahtangankan (\textit{non-transferable}). Penelitian terbaru menunjukkan bahwa SBT sangat efektif untuk merepresentasikan atribut kualitatif seperti reputasi, kredibilitas, atau sertifikasi keahlian akademik yang tidak boleh diperjualbelikan \cite{pinna2025soulbound}.
\end{itemize}

Penerapan SBT memungkinkan sistem memisahkan "modal ekonomi" (yang bisa dicairkan) dengan "modal sosial" (reputasi reviewer), mencegah terjadinya komodifikasi kualitas akademik yang berlebihan.


\section{Model Insentif Multi-Dimensi Berbasis Blockchain}
Pemanfaatan teknologi Blockchain tentu akan meningkatan rasa kepercayaan dan mendukung "single source of truth" yang transparan
 \cite{ihle2023incentivemechanisms} dan immutability yang tinggi.
Agar terwujud, Blockchain memerlukan sebuah model sistem insentif yang efektif, adil,aman , transparan dan berkelanjutan. Model insentif 
"multi-dimensi" diciptakan sebagai pendekatan yang menggabungkan insentif moneter dan non-moneter untuk menciptakan sistem yang berkelanjutan.

Bagian ini akan mengulas dua model insentif yang menjadi landasan perancangan sistem. STEEMIT, yang merepresentasikan keberhasilan manajemen aset hybrid dalam komunitas sosial terdesentralisasi dan
PubChain yang menawarkan quality control spesifik untuk bidang publikasi ilmiah
\subsection{Model Berbasis Aset Ganda (Dual-Token Economy)}
Sebagai salah satu pionir dalam blockchain-based online community, Steemit menawarkan kerangka kerja empiris mengenai bagaimana insentif kripto dapat memengaruhi partisipasi pengguna. 
Studi empiris menunjukkan bahwa partisipasi aktif pengguna dalam komunitas terdesentralisasi sangat dipengaruhi oleh persepsi mereka terhadap dua jenis modal: Social Capital (modal sosial) dan Share Capital (modal kepemilikan).

Meskipun STEEMIT tidak menawarkan algoritma teknis mendalam, penelitian ini memetakan logika konseptual penting mengenai "Economic Feedback" yang menjadi dasar distribusi insentif dalam sistem Steemit\cite{liu2022steemit}.
Mekanisme ini mendistribusikan token dari reward pool harian berdasarkan bobot suara (stake-weighted voting). Secara matematis, logika distribusi insentif ini dapat diformulasikan sebagai berikut:
\begin{equation}
R_i = 
\left(
    \frac{
        \text{Votes}_i \times \text{Weight}_i
    }{
        \sum_{\text{all}} (\text{Votes}_{\text{all}} \times \text{Weight}_{\text{all}})
    }
\right)
\times \text{TotalRewardPool}
\end{equation}
Di mana $R_i$ adalah insentif yang diterima pengguna, $\text{Votes}_i$ adalah 
jumlah dukungan yang diterima, dan $\text{Weight}_i$ merupakan kekuatan voting 
yang ditentukan oleh kepemilikan aset yang terkunci (vested token).

Yang utama dari model STEEMIT bagi penelitian ini berada di arsitektur Dual-Token yang mengadopsi aset hybrid. STEEMIT menerapkan pemisahan aset menjadi token 
likuid (STEEM/SBD) untuk pembayaran dan locked token (STEEM Power / SP) untuk hak voting dan kelola reputasi.
Kepemilikan Steem Power terbukti menciptakan "Psychological Ownership" yang mendorong pengguna untuk berkontribusi lebih aktif demi menjaga nilai jangka panjang platform \cite{liu2022steemit}.
Penelitian ini akan mengadopsi arsitektur token Steemit (pemisahan antara token dompet likuid dan token reputasi terkunci)
sebagai landasan perancangan crypto wallet untuk reviewer, guna menyeimbangkan kebutuhan finansial jangka pendek dengan komitmen kualitas jangka panjang.

\subsection{Model Ekonomi Sirkular dan Sidechain Pegging}
Berbeda dengan model STEEMIT yang berfokus pada interaksi sosial, PubChain menawarkan model kerja insentif yang dirancang spesifik untuk ekosistem publikasi ilmiah yang menuntut standar kualitas ketat. Tantangan utama dalam insentif akademik bukan sekedar memacu aktivitas, namun mengatasi penurunan kualitas peer review akibat kurangnya motivasi reviewer \cite{wang2020pubchain}.

Kontribusi fundamental dari PubChain bagi penelitian ini adalah pemanfaatan teknologi \textit{sidechain} dan mekanisme ekonomi sirkular untuk menjamin ketersediaan dana insentif. Mekanisme ini terdiri dari komponen-komponen berikut:

\begin{itemize}
    \item Mekanisme Two-Way Peg (Jembatan Aset): 
    PubChain tidak beroperasi sebagai \textit{blockchain} terisolasi, melainkan sebagai \textit{sidechain} yang terikat pada \textit{parent chain} utama (seperti Bitcoin atau Ethereum). Mekanisme ini memungkinkan perpindahan aset antar-rantai tanpa mengubah total suplai global melalui proses \textit{Locking} di rantai utama dan \textit{Minting} di rantai samping \cite{back2014pegged}. Hal ini vital untuk memberikan nilai intrinsik pada token insentif, sehingga reviewer menerima aset yang memiliki nilai tukar nyata, bukan sekadar poin digital.

    \item Sumber Pendanaan Hibrida (Fee dan Minting): 
    Untuk menjamin keberlanjutan insentif dan mengatasi masalah \textit{Cold Start}, PubChain menerapkan dua sumber pendanaan ke dalam \textit{Reward Pool}:
    \begin{enumerate}
        \item Submission Fee (X): Biaya yang dibayar penulis sebagai disinsentif untuk mencegah \textit{spamming} makalah berkualitas rendah.
        \item Inflationary Minting (Y): Kebijakan moneter di mana setiap blok baru otomatis mencetak token baru ($Y$) sebagai subsidi sistem.
    \end{enumerate}
    Dana gabungan dari Fee ($X$) dan Subsidi Minting ($Y$) inilah yang dikumpulkan dalam \textit{Pool} untuk diperebutkan oleh partisipan \cite{wang2020pubchain}.

    \item Penilaian Bobot Reputasi: 
    Tidak semua suara reviewer setara. Skor akhir jurnal $S_i$ dihitung menggunakan rata-rata terbobot, dimana bobotnya ditentukan oleh kualitas historis reviewer tersebut. Mekanisme diformulasikan sebagai berikut:
    \begin{equation}
        S_i = \sum_{j} W^{i,j} \cdot Z_{i,j}
    \end{equation}
    Di mana $Z_{i,j}$ adalah skor yang diberikan oleh reviewer, dan $W^{i,j}$ adalah bobot reputasi reviewer yang telah dinormalisasi berdasarkan penilaian pembaca. Pendekatan ini mencegah manipulasi skor oleh reviewer yang tidak kompeten \cite{wang2020pubchain}.

    \item Distribusi Berbasis Threshold: 
    Mekanisme yang bertujuan mencegah review "asal-asalan". PubChain menerapkan fungsi threshold kualitas $\lambda$. Reward hanya bisa diberikan jika skor kualitas $S_i$ melampaui $\lambda$. Mekanisme ini diformulasikan sebagai berikut:
    \begin{equation}
        G_i \propto \text{Pool} \times \frac{\max(S_i - \lambda,\, 0)}{\sum \text{Total Surplus}}
    \end{equation}
    Dalam model ini, fungsi $\max(S_i - \lambda, 0)$ memastikan bahwa kontribusi dengan kualitas di bawah standar ($\lambda$) akan menghasilkan nilai insentif nol, terlepas dari seberapa banyak pekerjaan yang dilakukan.
\end{itemize}

Penelitian ini akan mengadopsi mekanisme \textit{Minting} PubChain sebagai solusi subsidi "gaji dasar" dan algoritma distribusinya untuk memecahkan masalah \textit{principal-agent} dalam OJS \cite{squazzoni2013incentive}, menjamin anggaran hanya tersalur ke kontribusi substantif.


\subsection{Model Validasi Terpusat dan Dana Jaminan}
Sementara PubChain menawarkan solusi likuiditas, Eureka menawarkan alur kerja validasi yang lebih relevan untuk diadopsi ke dalam OJS, memecahkan masalah validasi kualitas dengan tetap mempertahankan peran Editor \cite{niya2019blockchain}.

\begin{itemize}
    \item Validasi Terpusat Editor (Oracle): 
    Dalam ekosistem Eureka, pencairan insentif tidak dilakukan secara otomatis berdasarkan \textit{voting} pembaca umum, melainkan melalui validasi Editor. Dana insentif tetap terkunci (\textit{escrow}) dalam \textit{Smart Contract} hingga Editor memverifikasi kualitas ulasan tersebut. Jika Editor menyetujui (\textit{approve}), kontrak tereksekusi dan token ditransfer ke dompet Reviewer. Mekanisme ini menjaga integritas akademik OJS di mana kualitas ulasan dinilai oleh ahli.

    \item Held-Back Funds (Dana Jaminan Mutu): 
    Eureka menerapkan mekanisme \textit{Time-Locked Smart Contract} untuk jaminan kualitas pasca-publikasi. Sebagian biaya submisi ditahan (\textit{held back}) dalam \textit{Smart Contract} untuk periode waktu tertentu. Dana ini berfungsi sebagai "uang pertaruhan"; jika dikemudian hari ditemukan plagiasi, dana hangus atau diberikan ke pelapor. Jika aman, dana dikembalikan ke penulis atau dicairkan sepenuhnya \cite{niya2019blockchain}.
\end{itemize}

Penelitian ini akan mengadopsi logika alur kerja Eureka untuk merancang interaksi antara OJS dan \textit{Smart Contract}, memastikan bahwa otomatisasi pembayaran tetap tunduk pada keputusan editorial manusia demi menjaga standar ilmiah.


\section{Sintesis Research Gap}

Berdasarkan tinjauan terhadap literatur dan studi komparatif pada sub-bab sebelumnya, penelitian ini mengidentifikasi tiga kesenjangan fundamental (\textit{research gaps}) yang menghambat penerapan langsung model insentif blockchain yang ada ke dalam ekosistem \textit{Open Journal System} (OJS).

\begin{enumerate}
    \item Kesenjangan Validasi (Crowd vs Expert): 
    Model \textit{PubChain} dan \textit{Steemit} mendasarkan validasi kualitas pada \textit{crowd-voting} (suara terbanyak dari pembaca umum). Pendekatan ini tidak kompatibel dengan OJS yang bersifat \textit{double-blind review} dan membutuhkan kepakaran spesifik \cite{wang2020pubchain}\cite{liu2022steemit}. OJS membutuhkan mekanisme di mana Editor bertindak sebagai \textit{Oracle} tunggal yang memicu \textit{Smart Contract}, mirip dengan logika \textit{Eureka}, namun tanpa menghilangkan transparansi algoritma pembagian insentifnya.
    
    \item Dilema Sumber Pendanaan (Fee vs Subsidi): 
    Terdapat dikotomi ekstrem dalam model pendanaan saat ini. Model \textit{Eureka} bergantung sepenuhnya pada biaya penulis (\textit{Submission Fee}), yang berpotensi menghambat partisipasi penulis dari negara berkembang (masalah \textit{Barrier to Entry}) \cite{niya2019blockchain}. Sebaliknya, model \textit{PubChain} mengandalkan inflasi token (\textit{Minting}), yang berisiko menurunkan nilai token jika tidak ada permintaan nyata. Belum ada model yang menyeimbangkan beban biaya penulis dengan subsidi sistem secara proporsional untuk menjaga keberlanjutan dana insentif di jurnal berskala kecil.
    
    \item Fragmentasi Manajemen Aset:
    Literatur mengenai manajemen aset kripto memisahkan token utilitas (uang) dan token reputasi (SBT) dalam studi yang berbeda \cite{pinna2025soulbound}. Belum ada arsitektur \textit{crypto wallet} terintegrasi untuk OJS yang mampu mengelola aset \textit{hybrid} ini secara simultan: memisahkan likuiditas untuk kebutuhan ekonomi reviewer sambil mengunci reputasi agar tidak dapat diperjualbelikan (mencegah komodifikasi akademik).
\end{enumerate}


Penelitian ini tidak bertujuan menciptakan blockchain baru, melainkan merancang \textit{middleware} (jembatan logika) yang mengisi kesenjangan tersebut. Solusi yang diusulkan adalah arsitektur sistem hibrida yang menggabungkan mekanisme \textit{Inflationary Minting} (dari PubChain) untuk subsidi dana, dengan logika \textit{Editor-as-Oracle} (dari Eureka) untuk validasi kualitas, yang dikelola dalam satu \textit{wallet} berbasis standar ganda (ERC-20 dan SBT).

