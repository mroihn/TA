% ------------------------------------------------------------------- %
%  LITERATURE REVIEW CHAPTER
%  AUTHOR: Raden Francisco Trianto Bratadiningrat (@NoHaitch)
%  DATE: 2025-11-11
% ------------------------------------------------------------------- %

% ---------- JUDUL BAB ----------
\chapter{Kajian Pustaka}

% ---------- PUBLIKASI ILMIAH ----------
\section{Sistem Publikasi Ilmiah}

Sistem publikasi ilmiah mulai berkembang sejak abad ke-17 dengan penerbitan 
\textit{Philosophical Transactions} oleh Royal Society. Sistem publikasi ilmiah 
terus berevolusi untuk menjawab kebutuhan validasi, standarisasi, dan kolaborasi 
antar ilmuwan dalam mendokumentasikan serta menyebarluaskan pengetahuan secara 
formal (\cite{fyfe2022historyscientificjournals}).

\subsection{Peran dalam Sistem Publikasi Ilmiah}

Secara umum, sistem publikasi ilmiah terdiri dari beberapa peran utama, yaitu 
(\cite{aliwatson2016peerreviewpublication, drozdz2024peerreviewprocess}):

\begin{enumerate}
    \item \textbf{Penulis (\textit{Author}): }\\
        Individu atau kelompok peneliti yang menyusun dan mengajukan manuskrip 
        ilmiah kepada jurnal. Penulis bertanggung jawab atas orisinalitas, 
        akurasi, dan penyempurnaan manuskrip berdasarkan masukan editor dan penelaah.
    
    \item \textbf{Editor (\textit{Editorial Board}): }\\
        Mengelola seluruh proses publikasi, termasuk penapisan manuskrip awal 
        (\textit{desk review}), pemilihan penelaah, evaluasi laporan penelaah, 
        dan pengambilan keputusan editorial akhir.
    
    \item \textbf{Penelaah (\textit{Peer Reviewer}): }\\
        Ahli bidang yang secara independen menilai manuskrip berdasarkan validitas 
        ilmiah, orisinalitas, metodologi, dan relevansi, serta memberikan 
        rekomendasi dan komentar untuk peningkatan kualitas manuskrip.
    
    \item \textbf{Tim Produksi (\textit{Production Team}): }\\
        Melakukan penyuntingan bahasa (\textit{copyediting}), pemformatan 
        (\textit{typesetting}), dan peninjauan akhir (\textit{proofreading}) 
        untuk memastikan manuskrip yang diterima memenuhi standar publikasi.
    
    \item \textbf{Penerbit (\textit{Publisher}): }\\
        Organisasi atau lembaga yang mengelola jurnal, menetapkan kebijakan 
        editorial, mengelola infrastruktur publikasi, serta mengurus distribusi 
        dan aksesibilitas karya ilmiah.
    
    \item \textbf{Pembaca (\textit{Reader}): }\\
        Pengguna akhir yang mengakses dan memanfaatkan karya ilmiah yang 
        dipublikasikan. Pembaca dapat memberikan umpan balik, kritik, atau 
        komentar pasca-publikasi untuk memperkaya ekosistem ilmiah yang terbuka.
\end{enumerate}

\subsection{Proses Sistem Publikasi Ilmiah}

Proses publikasi dapat dijelaskan sebagai serangkaian tahapan yang melibatkan kolaborasi antara penulis, editor, penelaah, dan penerbit. 
Sistem publikasi ilmiah terdiri dari beberapa proses yang dirangkum pada tabel~\ref{tab:proses-publikasi} 
(\cite{aliwatson2016peerreviewpublication, drozdz2024peerreviewprocess}).

\begin{xltabular}{\textwidth}{|c|l|X|}
\caption{Tahapan Proses Publikasi Ilmiah}\label{tab:proses-publikasi} \\

\hline
\rowcolor[HTML]{EFEFEF}
\textbf{No.} & \textbf{Tahapan Proses} & \textbf{Penjelasan} \\
\hline
\endfirsthead
\multicolumn{3}{c}{{\tablename\ \thetable{} Tahapan Proses Publikasi Ilmiah}} \\
\hline
\rowcolor[HTML]{EFEFEF}
\textbf{No.} & \textbf{Tahapan Proses} & \textbf{Penjelasan} \\
\hline
\endhead

\hline
\multicolumn{3}{r}{{Bersambung ke halaman berikutnya}} \\
\endfoot

\hline
\endlastfoot

1 & \textbf{\textit{Manuscript Preparation}} & Penulis menyusun manuskrip sesuai panduan dan standar jurnal tujuan. \\
\hline
2 & \textbf{\textit{Initial Submission}} & Manuskrip dikirimkan oleh penulis ke jurnal untuk dipertimbangkan proses seleksi. \\
\hline
3 & \textbf{\textit{Editorial Screening}} & Editor menilai cakupan, orisinalitas, dan kelayakan dasar manuskrip; manuskrip yang tidak memenuhi kriteria dapat ditolak pada tahap awal (\textit{desk rejection}). \\
\hline
4 & \textbf{\textit{Peer Review Selection}} & Editor memilih dan mengundang penelaah ahli yang relevan. \\
\hline
5 & \textbf{\textit{Peer Review Evaluation}} & Penelaah secara independen menilai kualitas, metodologi, dan kontribusi ilmiah manuskrip. \\
\hline
6 & \textbf{\textit{Reviewer Recommendations}} & Penelaah memberikan rekomendasi penerimaan, revisi, atau penolakan. \\
\hline
7 & \textbf{\textit{Editorial Decision}} & Editor menentukan keputusan akhir berdasarkan hasil penelaahan. \\
\hline
8 & \textbf{\textit{Decision Communication}} & Editor menyampaikan keputusan dan umpan balik kepada penulis. \\
\hline
9 & \textbf{\textit{Author Revisions}} & Penulis melakukan revisi berdasarkan umpan balik yang diberikan. \\
\hline
10 & \textbf{\textit{Resubmission}} & Manuskrip revisi dikirimkan kembali oleh penulis. \\
\hline
11 & \textbf{\textit{Re-review Process}} & Editor dan/atau penelaah menilai ulang revisi manuskrip. \\
\hline
12 & \textbf{\textit{Final Decision}} & Editor membuat keputusan akhir terhadap manuskrip. \\
\hline
13 & \textbf{\textit{Production/Copyediting}} & Tim produksi menyunting bahasa, memformat, dan memeriksa naskah sebelum terbit. \\
\hline
14 & \textbf{\textit{Publication}} & Artikel dipublikasikan secara resmi, daring atau cetak, dengan DOI. \\
\hline
15 & \textbf{\textit{Post-Publication}} & Pembaca dapat memberi umpan balik atau komentar paska-publikasi. \\
\end{xltabular}


% ---------- PEER REVIEW ----------
\section{\textit{Peer Review}}

\textit{Peer Review} adalah proses evaluasi manuskrip oleh para ahli sejawat (\textit{peer}), yaitu pakar dalam bidang ilmu yang relevan.
Proses ini bertujuan untuk menilai kualitas, validitas, orisinalitas, dan kontribusi ilmiah dari karya sebelum diterbitkan secara resmi (\cite{aliwatson2016peerreviewpublication, drozdz2024peerreviewprocess}). 
Para penelaah melakukan evaluasi secara independen dan memberikan rekomendasi serta masukan untuk peningkatan kualitas manuskrip (\cite{aliwatson2016peerreviewpublication}).


% ---------- PEER REVIEW MODELS ----------
\subsection{Model \textit{Peer Review}}

Dalam sistem publikasi ilmiah, terdapat berbagai model \textit{peer review} yang telah diterapkan maupun sedang dikembangkan. 
Beberapa model peer review dirangkum dan dipaparkan secara sistematis pada Tabel~\ref{tab:peer-review-models} (\cite{drozdz2024peerreviewprocess, aliwatson2016peerreviewpublication}).

\begin{xltabular}{\textwidth}{|c|l|X|}
\caption{Model \textit{Peer Review}}\label{tab:peer-review-models} \\
\hline
\rowcolor[HTML]{EFEFEF}
\textbf{No.} & \textbf{Model \textit{Peer Review}} & \textbf{Penjelasan} \\
\hline
\endfirsthead
\multicolumn{3}{c}{{\tablename\ \thetable{} Model \textit{Peer Review}}} \\
\hline
\rowcolor[HTML]{EFEFEF}
\textbf{No.} & \textbf{Model \textit{Peer Review}} & \textbf{Penjelasan} \\
\hline
\endhead

1 & \textbf{\textit{Closed Peer Review}} & Model dimana identitas pihak-pihaknya (penulis, penelaah, atau editor) dirahasiakan. Mencakup single-blind, double-blind, dan triple-blind review. \\
\hline
2 & \textbf{\textit{Single-Blind Review}} & Penelaah mengetahui identitas penulis, namun penulis tidak mengetahui identitas penelaah untuk mengurangi bias penulis. \\
\hline
3 & \textbf{\textit{Double-Blind Review}} & Identitas penulis dan penelaah disembunyikan dari masing-masing pihak untuk meminimalkan bias kedua belah pihak. \\
\hline
4 & \textbf{\textit{Triple-Blind Review}} & Identitas penulis, penelaah, dan editor disembunyikan untuk mengurangi bias secara menyeluruh dalam proses evaluasi. \\
\hline
5 & \textbf{\textit{Open Peer Review}} & Identitas pihak-pihaknya diketahui bersama tanpa dirahasiakan, dengan hasil review yang dapat dipublikasikan untuk meningkatkan transparansi. \\
\hline
6 & \textbf{\textit{Transparent Peer Review}} & Laporan dan proses review dipublikasikan secara terbuka, meskipun identitas penelaah dapat tetap dirahasiakan. \\
\hline
7 & \textbf{\textit{Collaborative Peer Review}} & Beberapa penelaah bekerja bersama secara kolaboratif untuk memberikan evaluasi yang lebih komprehensif terhadap manuskrip. \\
\hline
8 & \textbf{\textit{Pre-Publication Peer Review}} & Evaluasi manuskrip dilakukan sebelum publikasi sebagai tahap utama untuk menjamin kualitas karya ilmiah. \\
\hline
9 & \textbf{\textit{Post-Publication Peer Review}} & Penilaian dan komentar dilakukan setelah publikasi untuk memberikan umpan balik berkelanjutan pada karya ilmiah. \\
\hline

\end{xltabular}


% ---------- OPEN PEER REVIEW ----------
\subsection{\textit{Open Peer Review (OPR)}}

Menurut Ross-Hellauer, \textit{Open Peer Review} atau OPR didefinisikan sebagai istilah umum yang mencakup berbagai bentuk adaptasi model \textit{peer review} agar selaras dengan prinsip dan tujuan Sains Terbuka (\textit{Open Science}) (\cite{ross2017openpeerreview}).
\textit{Open Science} bertujuan untuk meningkatkan kualitas dan efisiensi penelitian melalui transparansi, reprodusibilitas, aksesibilitas, dan kolaborasi yang lebih luas (\cite{lahti2017openscience}).

Ross-Hellauer mendefinisikan sifat-sifat OPR pada Tabel~\ref{tab:peer-review-traits}~(\cite{ross2017openpeerreview}).

\begin{xltabular}{\textwidth}{|c|l|X|}
\caption{Sifat-Sifat \textit{Open Peer Review}}\label{tab:peer-review-traits} \\

\hline
\rowcolor[HTML]{EFEFEF}
\textbf{No.} & \textbf{Sifat \textit{Open Peer Review}} & \textbf{Penjelasan} \\
\hline
\endfirsthead

\multicolumn{3}{c}{{\tablename\ \thetable{} Sifat-Sifat \textit{Open Peer Review}}} \\
\hline
\rowcolor[HTML]{EFEFEF}
\textbf{No.} & \textbf{Sifat \textit{Open Peer Review}} & \textbf{Penjelasan} \\
\hline
\endhead

\hline
\multicolumn{3}{r}{{Bersambung ke halaman berikutnya}} \\
\endfoot

\hline
\endlastfoot  

1 & \textit{\textbf{Open identities}} & Penulis dan Penelaah saling mengetahui identitas satu sama lain. \\
\hline
2 & \textit{\textbf{Open reports}} & Laporan telaah dipublikasikan bersama artikel terkait yang ditelaah. \\
\hline
3 & \textit{\textbf{Open participation}} & Komunitas ilmiah umum dapat berkontribusi dalam proses penelaahan. \\
\hline
4 & \textit{\textbf{Open interaction}} & Diskusi timbal balik antara para penulis dan penelaah, didorong dan diperbolehkan. \\
\hline
5 & \textit{\textbf{Open pre-review manuscripts}} & Naskah harus sudah tersedia sebelum dimulainya prosedur formal \textit{Peer Review}. \\
\hline
6 & \textit{\textbf{Open final-version commenting}} & Versi publikasi final dapat ditelaah atau dikomentari oleh publik. \\
\hline
7 & \textit{\textbf{Open platforms (“decoupled review”)}} & Proses penelaahan difasilitasi oleh entitas organisasi yang berbeda dari entitas publikasi. \\
\end{xltabular}


% ---------- MASALAH DALAM PEER REVIEW ----------
\subsection{Masalah dalam \textit{Peer Review}}

Penggunaan \textit{Peer Review} dalam sistem publikasi ilmiah menghadapi berbagai tantangan. 
Masalah-masalah tersebut dapat dikelompokkan menjadi empat kategori utama (\cite{elguebaly2023peerreviewissues, tennant2020peerreviewissues, velterop2015peerreviewissues}):


\subsubsection{Anonimitas}

Sistem \textit{double-blind} atau \textit{single-blind peer review} dapat menciptakan perbedaan dalam gaya dan kualitas telaah, serta mempengaruhi akuntabilitas penelaah terhadap evaluasi yang diberikan (\cite{elguebaly2023peerreviewissues}). 
Sistem anonimitas juga berdampak pada transparansi proses evaluasi dan kepercayaan penulis terhadap keadilan penilaian (\cite{tennant2020peerreviewissues}).

\subsubsection{Sistem Pemilihan Penelaah}

\begin{enumerate}
    \item \textbf{Keterbatasan jumlah penelaah}: Jumlah penelaah ahli dalam bidang tertentu sangat terbatas. Akibatnya, penelaah yang sama sering diminta berkali-kali untuk melakukan telaah. Hal ini meningkatkan beban kerja penelaah dan menyebabkan kelelahan yang berdampak pada kualitas telaah (\cite{elguebaly2023peerreviewissues}).
    
    \item \textbf{Seleksi penelaah dan perspektif keilmuan}: Proses pemilihan penelaah jarang mempertimbangkan keberagaman perspektif dan latar belakang keilmuan. Akibatnya, telaah menjadi bias terhadap pendekatan atau metode tertentu (\cite{elguebaly2023peerreviewissues, tennant2020peerreviewissues}). Hal ini menyebabkan penelitian dengan pendekatan berbeda menjadi kurang diakui (\cite{velterop2015peerreviewissues}).
    
    \item \textbf{Tidak ada mekanisme pembelajaran penelaah}: Sistem \textit{peer review} tidak memiliki cara formal bagi penelaah untuk belajar dari pengalaman telaah mereka. Penelaah tidak mendapat kesempatan untuk meningkatkan kemampuan telaah mereka secara berkelanjutan (\cite{elguebaly2023peerreviewissues}).
\end{enumerate}

\subsubsection{Insentif \textit{Peer Review}}

\begin{enumerate}
    \item \textbf{Motivasi dan insentif yang kurang jelas}: Penelaah sering tidak memiliki motivasi yang cukup untuk melakukan \textit{peer review}. Mereka jarang mendapat kompensasi finansial atau pengakuan dalam karir akademik (\cite{elguebaly2023peerreviewissues}). Akibatnya, banyak penelaah memprioritaskan kepentingan karir pribadi daripada kepentingan ilmu (\cite{velterop2015peerreviewissues}).
    
    \item \textbf{Kurangnya pengakuan terhadap penelaah}: Kontribusi penelaah jarang diakui secara resmi dalam sistem akademik (\cite{elguebaly2023peerreviewissues}). Padahal, \textit{peer review} adalah pekerjaan penting yang menjaga kualitas dan kepercayaan penelitian (\cite{tennant2020peerreviewissues}).
\end{enumerate}

\subsubsection{Metrik Penilaian Hasil Publikasi dan Telaah}

\begin{enumerate}
    \item \textbf{Tekanan metrik penilaian jurnal}: Penilaian penelitian sering didasarkan pada faktor dampak (\textit{impact factor}) dan metrik jurnal lainnya. Tekanan ini menyebabkan proses \textit{peer review} menjadi terdistorsi dan mendorong hasil publikasi yang bias (\cite{elguebaly2023peerreviewissues}). Sistem ini juga menciptakan motivasi yang salah bagi penelaah dan editor (\cite{velterop2015peerreviewissues}).
    
    \item \textbf{Perubahan model bisnis penerbitan}: Perubahan dari model berlangganan tradisional menjadi akses terbuka (\textit{open access}) menciptakan ketidakpastian. Hal ini mempengaruhi keberlanjutan dan kebebasan proses \textit{peer review} (\cite{elguebaly2023peerreviewissues}) serta standar kualitas publikasi (\cite{tennant2020peerreviewissues}).
    
    \item \textbf{Munculnya jurnal dan konferensi predatori}: Beberapa jurnal dan konferensi menerima artikel tanpa melakukan \textit{peer review} yang baik. Hal ini mengancam kepercayaan terhadap sistem \textit{peer review} dan merusak kredibilitas metrik penilaian publikasi (\cite{elguebaly2023peerreviewissues}).
    
    \item \textbf{Kurangnya keberagaman perspektif dalam penilaian}: Sistem penilaian kurang mempertimbangkan berbagai perspektif dan pendekatan keilmuan. Akibatnya, penelitian dengan pendekatan berbeda menjadi kurang dihargai dalam komunitas ilmiah (\cite{elguebaly2023peerreviewissues, tennant2020peerreviewissues}).
\end{enumerate}

